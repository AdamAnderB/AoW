\section{Discussion}
Alone, eye-fixations are meaningless. Exacting Meaning for x- and y-coordinates is achieved
through time, visual stimuli, and audio stim. These \textit{four core} constructs correspond directly with the variables of our experiments, research questions, and data analyses. However, managing these constructs is complex. Data wrangling, through lines of code, knits these constructs together, gradually constructing bridges of understanding.

vwp- four core
meaning is built/mapped/poured into:
-eye fixations
through data wrangling of:
-visual stimuli
-audio sitmuli
-time
the data wrangling cycle: is an iterative cycle of exploration that allows the research to ask and answer practical questions (e.g., who to remove, what eye-fixation to keep, how to join data without losing data, How to build robust models that provide insight into the data). Here, we used it to answer the following questions:


Decisions for the researcher to make:
- what are the exclusion criteria for participants?
These types of removal should always always be made before hand. If you start removing people here based on feeling it doesn't feel good for science and it doesn't feel good yourself. 


- what are the behavioral task checks and what are the criteria for removal?
Removal here can be strict. However, beyond setting a strict limit. The more important thing is to report what criterion you used for removal not simply that you removed certain items and participants for 'technology' failures or whatever it is that is in a lot of papers.


- how to identify and select quadrants
Lab-based eye-tracking is extremely precise, web-based eye-tracking is more variable. This is part of the reason that it is difficult to wrangle. Previous web-based eye-tracking studies have shown that eye-fixation variability can be captured even with strict standards for removal. That is, eye-fixations outside the target areas in figure \ref{fig:signal_noise} are excluded regardless of how close they are to the area (i.e., classifying web-based eye-fixation the same way that lab-based eye tracking does). In our data, we looked to examine where eye-fixations are distributed throughout the trial.

- frame-rate cutoffs
Basically below 5 seems to be "unusable". However, beyond that, it really depends on what your RQ is. If you are looking at the verb\_type effect this is found all the way down to 6-10 fps. However, this is only true with the native speaker is speaking. The effect in the non-native speaker cannot be captured until around 12-17 Hzs

- bin sizes
Perhaps a remnant from the past. It no longer seems necessary outside of visualizations. Basically the guidance here is that you must understand that increasing the bin size decreases the amount of bins. e.g., if you have 250ms and you want 50 binwidth bins then you will only  5 bins. Or the opposite.


data wrangling is pervasive throughout the data analysis process from raw to visualizing model results.

Data wrangling is a method of exploration that allows us to gain a fimiliarity with the constructs that we think about. Improving our experimental design, data analysis, data visualization, and happiness.

\section{Conclusions}
Data 
Web-based eye-tracking is here to stay. With that comes a demand for mastering data-wrangling skills. Currently, However, The opportunities that it offers are only beginning to be understood. 

However, for the first time, online the tools of data wrangling  have provided the research with first hard access to the real data.  





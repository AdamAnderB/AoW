\section{Abstract}

Web-based eye-tracking is more accessible than ever. Researchers can finally carry out visual world paradigm studies remotely and access never before tested populations via the internet without the need for an expensive eye-tracker. Web-based eye-tracking, however, requires careful planning of the experiment and extensive data wrangling skills. In this paper, we provide a guide for building reproducible, open science eye-tracking studies using the online experiment builder Gorilla. We provide step-by-step instructions to building a typical linguistics eye-tracking study, and walk the reader through a series of data wrangling steps needed to prepare the data for visualization and analysis using the open source software environment, R. Importantly, we highlight the key decisions researchers need to make--and report--in order to reproduce an analysis. We demonstrate our approach by successfully carrying out a single change replication of an in-person eye-tracking study, \textcite{Porretta_et_al_2020}. We conclude with best practices and recommendations for future web-based eye-tracking studies.


\section{Keywords}
Data quality, online research, Open science, eye-tracking, psycholinguistics
\newpage



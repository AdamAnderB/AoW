\section{Conclusions}
Web-based eye-tracking is here to stay. With that comes a demand for mastering data-wrangling skills. Currently, However, The opportunities that it offers are only beginning to be understood. For the first time, researchers anywhere can design an experiment, implement their analysis, and share their results openly for the cost of participant payment. Additionally, the choices of how data is treated is now up to us, which leads to a need for widespread adoption of standardized practices throughout the data analysis process beginning with experiment design and throughout all data analysis. As shown here and by \textcite{Prystauka_Altmann_Rothman_2023} and \textcite{Vos_2017}, web-based eye-tracking is a powerful and accessible tool. Its convenience, cost, and reliability make it an easy choice for any researcher. However, the data wrangling is extensive. Here, we hope that this barrier has now been removed through our step-by-step Gorilla guide, replication of \textcite{Porretta_et_al_2020}, and our guide through the wilds of data wrangling in the \textit{\textbf{Art of Wrangling}}.
 
\section{Data availability statement}
All data and scripts are available through OSF. All data is within the data folder of the OSF stored repository. All scripts are linked through Github. The primary script for data wrangling and analysis is \inlineR{AOW\_r\_work\_flow.Rmd}: \url{https://osf.io/9zgd3/?view_only=d54c9ddd28944a49a16d600c43f661f0}

\section{Acknowledgements}
Provide acknowledgments at the end of the article before the references. Funding information may be provided in the relevant field during online submission. 

\section{Competing interests declaration}
The authors declare none.

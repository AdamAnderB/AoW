\section{Discussion}

\subsection{Eye-fixations}
Lab-based eye-tracking is extremely precise, web-based eye-tracking is more variable. This is part of the reason that it is difficult to wrangle. Previous web-based eye-tracking studies have shown that eye-fixation variability can be captured even with strict standards for removal. That is, eye-fixations outside the target areas in figure \ref{fig:signal_noise} are excluded regardless of how close they are to the area (i.e., classifying web-based eye-fixation the same way that lab-based eye tracking does). In our data, we looked to examine where eye-fixations are distributed throughout the trial. 




some points to hit:


online et is here to stay. So deal with it

vwp- four core
meaning is built/mapped/poured into:
-eye fixations
through data wrangling of:
-visual stimuli
-audio sitmuli
-time

Decisions for the researcher to make:
- what are the exclusion criteria for participants?
- what are the behavioral task checks and what are the criteria for removal?
- how to identify and select quadrants
- frame-rate cutoffs



the data wrangling cycle: 
data wrangling is iterative
data wrangling is about practically answer questions. Which participants need to be removed-
what data do I have/need
what tools do I need to accomplish that
implement
and, use

data wrangling is pervasive throughout the data analysis process from raw to visualizing model results.

Data wrangling is a method of exploration that allows us to gain a fimiliarity with the constructs that we think about. Improving our experimental design, data analysis, data visualization, and happiness.

\section{Conclusions}
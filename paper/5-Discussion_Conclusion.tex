\section{Discussion}

 

\subsection{Bin size}
Perhaps a remnant from the past. It no longer seems necessary outside of visualizations

\subsection{Frame rates}




vwp- four core
meaning is built/mapped/poured into:
-eye fixations
through data wrangling of:
-visual stimuli
-audio sitmuli
-time

Decisions for the researcher to make:
- what are the exclusion criteria for participants?
These types of removal should always always be made before hand. If you start removing people here based on feeling it doesn't feel good for science and it doesn't feel good yourself. 


- what are the behavioral task checks and what are the criteria for removal?
Removal here can be strict. However, beyond setting a strict limit. The more important thing is to report what criterion you used for removal not simply that you removed certain items and participants for 'technology' failures or whatever it is that is in a lot of papers.


- how to identify and select quadrants
Lab-based eye-tracking is extremely precise, web-based eye-tracking is more variable. This is part of the reason that it is difficult to wrangle. Previous web-based eye-tracking studies have shown that eye-fixation variability can be captured even with strict standards for removal. That is, eye-fixations outside the target areas in figure \ref{fig:signal_noise} are excluded regardless of how close they are to the area (i.e., classifying web-based eye-fixation the same way that lab-based eye tracking does). In our data, we looked to examine where eye-fixations are distributed throughout the trial.

- frame-rate cutoffs
Basically below 5 seems to be "unusable". However, beyond that, it really depends on what your RQ is. If you are looking at the verb/_type effect this is found all the way down to 6-10 fps. However, this is only true with the native speaker is speaking. The effect in the non-native speaker cannot be captured until around 12-17 Hzs


the data wrangling cycle: 
data wrangling is iterative
data wrangling is about practically answer questions. Which participants need to be removed-
what data do I have/need
what tools do I need to accomplish that
implement
and, use

data wrangling is pervasive throughout the data analysis process from raw to visualizing model results.

Data wrangling is a method of exploration that allows us to gain a fimiliarity with the constructs that we think about. Improving our experimental design, data analysis, data visualization, and happiness.

\section{Conclusions}

Web-based eye-tracking is here to stay. whereas in person eye-tracking is expensive, but often uses pre-processing software thatThe opportunities that it offers are only beginning to be understood. tools that it offers 
Data wrangling is a tool that 
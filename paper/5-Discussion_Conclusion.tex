\section{Discussion and Conclusions}

\subsection{Eye-fixations}
Lab-based eye-tracking is extremely precise, web-based eye-tracking is more variable. This is part of the reason that it is difficult to wrangle. Previous web-based eye-tracking studies have shown that eye-fixation variability can be captured even with strict standards for removal. That is, eye-fixations outside the target areas in figure \ref{fig:signal_noise} are excluded regardless of how close they are to the area (i.e., classifying web-based eye-fixation the same way that lab-based eye tracking does). In our data, we looked to examine where eye-fixations are distributed throughout the trial. As displayed in figure \ref{fig:signal_noise}, fixations are mostly distributed at the center of the screen, indicating no looks to quadrants. Whereas this remains true for competitor items throughout the trial, target items begin to move toward visual stimuli as early as the verb onset and much more extreme in later time frames. Crucially, however, the fixations do not always reach the actual quadrants. In analyzing the data from the shinny app, removing data between the center point of the screen and the inner-edges of the quadrants results in \~83.33\% data loss, which is more than twice as high as previous reports on two image stimuli web-based studies \parencite{Vos_2017}. However, if we move to a more relaxed categorization of simply splitting the screen into four quadrants, then only 6.71\% of data is lost. In contrast, maximal outer-edge removal results in very little data loss (max \~32\%). When removing inner-edge eye-fixations, the choice comes down to removing signal to avoid noise in spatial ambiguity, or embarrassing noise to maximally retain the signal. As shown in figure selection of Visual Stimuli competitors of \ref{fig:signal_noise} noise is randomly dist across quadrants similarly to the way they are distributed early in the trial before eye-movements tend toward visual stimuli. Here, we aim to strike the balance of the signal noise trade off by removing most of the data outside the screen size, maximally retaining inner data that shows trends. This leads us to believe that no bias would occur even if classifying data from the x,y fixation center (0.5,0.5). 




some points to hit:


online et is here to stay. So deal with it

vwp- four core
meaning is built/mapped/poured into:
-eye fixations
through data wrangling of:
-visual stimuli
-audio sitmuli
-time

Decisions for the researcher to make:
- what are the exclusion criteria for participants?
- what are the behavioral task checks and what are the criteria for removal?
- how to identify and select quadrants
- frame-rate cutoffs



the data wrangling cycle: 
data wrangling is iterative
data wrangling is about practically answer questions. Which participants need to be removed-
what data do I have/need
what tools do I need to accomplish that
implement
and, use

data wrangling is pervasive throughout the data analysis process from raw to visualizing model results.

Data wrangling is a method of exploration that allows us to gain a fimiliarity with the constructs that we think about. Improving our experimental design, data analysis, data visualization, and happiness.